\documentclass{beamer}
\usecolortheme{beaver}
\beamertemplatenavigationsymbolsempty
\setbeamertemplate{blocks}[rounded=true, shadow=true]
\setbeamertemplate{footline}[page number]
\setbeamertemplate{caption}{\raggedright\insertcaption\par}
%
\usepackage[utf8]{inputenc}
\usepackage[english, russian]{babel}
\usepackage{amssymb,amsfonts,amsmath,mathtext}
%\usepackage{subfig}
\usepackage[all]{xy} % xy package for diagrams
\usepackage{array}
\usepackage{multicol}% many columns in slide
\usepackage{hhline}%tables
\usepackage{multirow}
\usepackage{graphicx}
\graphicspath{{figures/}}  
\usepackage{wrapfig}%Обтекание фигур (таблиц, картинок и прочего)

% Your figures are here:

\title[Обратные задачи моделирования nPDE] %optional
{Обратные задачи в моделировании нейронных дифференциальных уравнений в частных производных}


\author[А. Терентьев] % (optional)
{Александр Терентьев}

\institute[MIPT] % (optional)
{
  
  Московский физико-технический институт, 
  
  Физтех-школа прикладной математики и информатики\\

  Кафедра интеллектуальных систем

  Научный руководитель: д.ф .- м.н. Стрижов Вадим Викторович

}



% 右下角使用logo的方式
% \logo{\includegraphics[height=1cm]{overleaf-logo}} 

%End of title page configuration block
%------------------------------------------------------------



%------------------------------------------------------------
%The next block of commands puts the table of contents at the 
%beginning of each section and highlights the current section:
%------------------------------------------------------------


\begin{document}

%The next statement creates the title page.
\frame{\titlepage}





\begin{frame}
\frametitle{Классификация траекторий динамических систем}
\begin{block}{Проблема}
В задачах EEG трудность вызывает получение точного сигнала от головного мозга. Исследователи встречаются со следующими проблемами. Высокая чувствительность прибора к движениям и тремору, обусловленному психоэмоциональным напряжением пациента, вызывает помехи в работе, что может затруднить диагностику. 
\end{block}
\begin{block}{Цель}
Целью работы является предложить метод решения восстановления источников сигнала ЭЭГ и уменьшения уровня шума в их определении.
Предлагается использовать физико-информированный подход в восстановлении, использующийся в задачах восстановления временных рядов, вносящий априорные знания о модели для уменьшения уровня шума от данных.
\end{block}

\end{frame}

\begin{frame}
\frametitle{Постановка обратной задачи}

\begin{block}{Дано}
\begin{enumerate}
    \item $\mathcal{D} = \{X_i\}_{i=1}^N$ - набор данных пространственно-временных рядов ЭЭГ, где $X_i = \chi(r, t): \mathbb{R}^(3\times1) \to \mathbb{R}^{K}$ - пространственно-временной ряд сигналов $K$.
    \item $\mathcal{S} = \{s_i(t)\}_{i=1}^M$ - конечный набор источников сигналов. 
\end{enumerate}
\end{block}
\begin{block}{Найти}
\begin{enumerate}
    \item $D(\hat{s}|X, \mathcal{D})$ - суперпозиция пространственно-временных рядов ЭЭГ $X_i$ по $M$ источникам

    \item $G(\hat{X}(t+1)|X(t), \hat{s}, \mathcal{D})$ - восстановление пространственно-временных рядов ЭЭГ $X_i$ на основе $s$ источника
\end{enumerate}
\end{block}
\begin{block}{Критерий}
MSE - ошибка предсказаний рядов $\|\hat{X} - {X}\|_2$
\end{block}
\end{frame}

\begin{frame}
\frametitle{О схеме восстановления источников сигналов}
\begin{figure}[H]
\centerline{
    \xymatrix{
    \chi_i: (N_x \times N_y \times N_z \times T)  \ar@{->}[rr]^-{D(\hat{s}|X, \mathcal{D})} && s \ar@{->}[rr]^-{G(\hat{X}|X(t), \hat{s}, \mathcal{D})} && \hat{\mathbf{X}}
    }
}
\caption{Метод Inverse NPDE}
\label{fig: LNN}
\end{figure}

\begin{figure}[!tbp]
 \centering
 \begin{minipage}[b]{0.4\textwidth}
     \includegraphics[width=\textwidth]{head_surface_mesh.jpg}
     \caption{Поверхность головы $\Gamma$ }
     \label{fig: head_surface_mesh}
 \end{minipage}
 \hfill
 \begin{minipage}[b]{0.4\textwidth}
     \includegraphics[width=\textwidth]{head_mesh.jpg}  
     \caption{Разрез объема головы $\Omega$ }
     \label{fig: head_surface_mesh}
 \end{minipage}
\end{figure}

\end{frame}

\begin{frame}
\frametitle{Воостановление электромагнитных потенциалов}
\begin{block}{Уравнения Максвелла в СГС}
\begin{columns}[c] % [c] — для вертикального центрирования содержимого
    \column{0.5\textwidth}
          \begin{enumerate}
            \item $\nabla\cdot\mathbf{A} + \frac{\varepsilon\mu}{c}\,\frac{\partial \varphi}{\partial t} = 0$
            \item $\square\varphi = -4\pi\,\frac{\rho}{\varepsilon}$
            \item $\square\mathbf{A} = -\frac{4\pi}{c}\,\mu\,\mathbf{j}$
            \item $\square = \Delta -\frac{\varepsilon\mu}{c^2}\frac{\partial^2}{\partial t^2}$
         \end{enumerate}
    \column{0.5\textwidth}
         \begin{figure}[!tbp]
             \includegraphics[width=0.4\textwidth]{head_mesh.jpg}  
             \caption{Разрез объема головы $\Omega$ }
        \end{figure}
  \end{columns}
\end{block}

\begin{block}{Задача nPDE}
    Необходимо восстановить $\rho, \phi, \mathbf{A}, \mathbf{j}$
    Моделируется нейросетью с 8 выходами. Граничное условие: $\phi|_\Gamma = 0,\mathbf{A}|_\Gamma = 0$
\end{block}

\begin{block}{Критерий}
    $\text{Loss} = \text{BCSLoss} + \text{PDELoss} + \|\hat{X} - {X}\|_2 + R(\mathbf{w})$
\end{block}
\end{frame}

\begin{frame}
\frametitle{Энергия в качетсве регуляризации}
\begin{block}{Идея}
    \begin{enumerate}
        \item Требуется добиться наиболее простое распределение зарядов. Обычные методы регуляризации не решают данную задачу.
        \item Предлагается в качестве регуляризации брать энергию электрического поля
    \end{enumerate}
\end{block}
\begin{block}{Подсчет энергии}
    \begin{enumerate}
        \item $\frac{d\mathcal{E}}{dt}(\mathbf{r}, t) = \phi(\mathbf{r}, t)$
        \item Граничное условие $\mathcal{E}(\mathbf{r} =-\infty) = 0$
    \end{enumerate}
\end{block}
\begin{block}{Критерий}
    $\text{Loss} = \text{BCSLoss} + \text{PDELoss} + \|\hat{X} - {X}\|_2 - \mathcal{E} + \text{BCSLoss}_\mathcal{E} + \text{PDELoss}_\mathcal{E}$
\end{block}
\end{frame}

\begin{frame}
\frametitle{Регуляризация: связь энергии и произведения $\phi\rho$}
\begin{block}{Лемма (о связи энергии и произведения потенциала на плотность заряда)}
Пусть $\Omega\subset\mathbb{R}^d$ с конечным объёмом, $\phi\rho\in L^1(\Omega)$, и пусть
$r_1,\dots,r_K\stackrel{\text{i.i.d.}}{\sim}\mu$ с плотностью $p(r)>0$ п.в.
Пусть
\[
E=\int_\Omega \phi(r)\rho(r)\,dr,\qquad
\hat E_K=\frac{1}{K}\sum_{i=1}^K \frac{\phi(r_i)\rho(r_i)}{p(r_i)}.
\]
Тогда $\mathbb{E}[\hat E_K]=E$ и $\hat E_K\to E$ почти наверное при $K\to\infty$.
Если дополнительно $\phi\rho/p\in L^2(\mu)$, то
\[
\sqrt{K}\,(\hat E_K-E)\xrightarrow{d}\mathcal{N}(0,\sigma^2),\quad
\sigma^2=\mathrm{Var}_\mu\!\Big(\frac{\phi\rho}{p}\Big).
\]
В частности, регуляризатор $\lambda E$ приближён выборочной регуляризацией $\lambda\hat E_K$ с погрешностью $O_p(1/\sqrt{K})$.
\end{block}
\end{frame}

\begin{frame}
\frametitle{Эксперимент 1: реальные данные (Brunton)}
\begin{block}{Данные}
Использованы реальные данные \texttt{brunton_uw_bio} (см. репозиторий). Сравнивались наш метод и метод Loretta.
\end{block}
\begin{block}{Результат}
На датчиках MSE восстановления потенциалов у обоих методов примерно одинаковая (см. таблицу — заполните значения).
\end{block}
\begin{table}[ht]
\centering
\begin{tabular}{l c c}
\hline
Метод & Sensor MSE & Примечания \\
\hline
Наш метод & --- & \\
Loretta & --- & \\
\hline
\end{tabular}
\caption{Сравнение MSE восстановления потенциала на датчиках (заполните данные)}
\end{table}
\end{frame}

\begin{frame}
\frametitle{Эксперимент 2: синтетические данные}
\begin{block}{Модель}
Смоделирована шапочка EEG на овальной голове, дискретизация 50~Гц, 2 источника сигналов.
\end{block}
\begin{block}{Результат}
Точность на датчиках — примерно одинаковая, но в глубине головы наше восстановление точнее (см. таблицу).
\end{block}
\begin{table}[ht]
\centering
\begin{tabular}{l c c c}
\hline
Метод & Sensor MSE & Deep MSE & Примечания \\
\hline
Наш метод & --- & --- & \\
Loretta & --- & --- & \\
\hline
\end{tabular}
\caption{Сравнение по датчикам и глубине (заполните данные)}
\end{table}
\end{frame}

\begin{frame}
\frametitle{Синтетические данные — визуализация}
\begin{figure}[ht]
\centering
\includegraphics[width=0.8\textwidth]{../figures/synthetic_head_views.png}
\caption{Вид синтетической головы и расположение датчиков: 3D вид (верхний левый) и проекции спереди/сверху/боковой.}
\end{figure}
\end{frame}

\begin{frame}
\frametitle{Выводы и планы на будущее}
\begin{enumerate}
    \item Сравнивать методы только по MSE на датчиках нецелесообразно — лучше оценивать восстановленные источники по задачам: регрессии или классификации.
    \item В перспективе: завершить теорию регуляризации, исследовать дополнительные регуляризаторы (не только энергия), выполнить сравнение на побочных задачах и расширить набор методов для регуляризации.
\end{enumerate}
\end{frame}

\begin{frame}
\frametitle{Схожие работы}
\begin{enumerate}
    \item Zhen Qi, Gregory M. Noetscher, Alton Miles, etc.,  Enabling electric field model of microscopically realistic brain
\end{enumerate}
\end{frame}

\end{document}

