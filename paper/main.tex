     %\documentclass[12pt, twoside]{article}
%\documentclass[12pt]{article}%REMOVE AFTER
\documentclass[sn-mathphys-num]{bst/sn-jnl}
\usepackage{a4wide}%REMOVE AFTER

%\usepackage{jmlda}
\usepackage[title]{appendix}%
\usepackage[utf8]{inputenc}
\usepackage[english, russian]{babel}
\usepackage[T2A]{fontenc}
\usepackage{lineno}
\usepackage{amssymb,amsfonts,amsmath,mathtext}
\usepackage{amsthm}
\usepackage{array}
%\usepackage{theorem}
\usepackage[all]{xy}
\usepackage{array}
\usepackage{multicol}
\usepackage{hhline}
\usepackage{graphicx}
\usepackage{float}
\usepackage{subcaption}
\usepackage{wrapfig}%Обтекание фигур (таблиц, картинок и прочего)
\usepackage{multirow}
\usepackage{pgfplots}
\pgfplotsset{compat=1.9}
\usepackage{graphicx}%
\usepackage{mathrsfs}%
\usepackage[title]{appendix}%
\usepackage{xcolor}%
\usepackage{textcomp}%
\usepackage{manyfoot}%
%\usepackage{booktabs}%
\usepackage{algorithm}%
\usepackage{algorithmicx}%
\usepackage{algpseudocode}%
\usepackage{listings}%
\graphicspath{{figures/}}  
\renewcommand{\thesubfigure}{\textit{\alph{subfigure}}}
\newcommand{\hdir}{.}

\theoremstyle{thmstylethree}
\newtheorem{definition}{Defenition}
\theoremstyle{thmstyletwo}
\newtheorem{remark}{Remark}

\theoremstyle{thmstyleone}
\newtheorem{theorem}{Theorem}
\newtheorem{lemma}{Lemma}

\raggedbottom
\begin{document}
%\English

\title
	[Inverse problems in nPDE] % short title for page headings, not necessary if a full title fits the headings
    {Inverse problems in modeling neural partial differential equations} % full title
\author*[1]{\fnm{V.} \sur{Strizhov}}\email{sasha-terentev@outlook.com}
\author*[1]{\fnm{A.} \sur{Terentyev}}\email{strijov@forecsys.ru}

\affil*[1]{\orgname{Forecsys LLC}, \orgaddress{\city{Moscow}, \country{Russia}}}

\abstract
{В задачах EEG трудность вызывает получение точного сигнала от головного мозга. Исследователи встречаются со следующими проблемами. Высокая чувствительность прибора к движениям и тремору, обусловленному психоэмоциональным напряжением пациента, вызывает помехи в работе, что может затруднить диагностику. Такие изменения называются артефактами движения. А если встраивать датчики напрямую в кору мозга, то они уже сами начинают влиять на его работу, а через некоторое время приходят в негодность из-за перестройки мозга вокруг них. Поэтому стоит задача точного определения восстановления местоположения источников сигнала, при условии высокого уровня шума для работы с этими источниками в медицинских целях.
Целью работы является предложить метод решения восстановления источников сигнала ЭЭГ и уменьшения уровня шума в их определении.
Предлагается использовать физико-информированный подход в восстановлении, использующийся в задачах восстановления временных рядов, вносящий априорные знания о модели для уменьшения уровня шума от данных.


\textbf{Keywords}: \emph{physical system; eeg; pde; }}

%these fields are filled in by the journal editors
%\doi{10.21469/22233792}
%\receivedEng{January 01, 2017}

\maketitle
%\linenumbers

\section{Введение}

Обратная задача ЭЭГ имеет фундаментальное значение для нейронауки, поскольку она
позволяет получить представление о пространственной и временной активности мозга при выполнении
различных задач. Точное решение задачи визуализации нейронных источников.
Проблема может способствовать пониманию внутренней работы
мозга и выявления областей с аномалиями проводимости, которые могут указывать на поврежденную ткань \cite{PascualMarqui1999}. Обратная задача ЭЭГ является плохо обусловленной задачей, поэтому у нее не существует единственного
решения. Чтобы восстановить приближенное решение, необходимы методы регуляризации по аналогии с такими методами MNE \cite{Grech2008} и томографии электрической активности низкого разрешения (LORETA). Которые учитывают взаимосвязь между
источниками тока и измеренными потенциалами в предположении квази
статического приближения, выраженного матрицей поля \cite{Weinstein1999}. Целью работы является предложить такую регуляризацию для обратной задачи nPDE, которая бы предлагала максимально простую физическую модель для описания источника поля.
В большинстве моделей используется предположение о квазистатичности процессов, происходящих в мозге. В общем случае это не верно. Для моделирования источников коротких сигналов в мозге необходимо учитывать зависимость поля от скорости изменения источников сигнала. Целью работы явлется предложить такую модель, которая автоматически учитывала зависимость поля от скорости изменения источников сигнала. Таким образом учитывая эффект индукции от резкого изменения токов. 
\begin{itemize}
    \item Исследовать существующие методы решения обратной задачи ЭЭГ 
    \item Предложить общий метод непрерывной обратной задачи ЭЭГ с помощью нейронных сетей, учитывающий электромагнитую природу задачи.
    \item Предложить метод регуляризации задчи, учитывающую физику. (Возможно стоит подумать в сторону энергии) 
    \item Предложить метод, учитывающий зависимость во времени
\end{itemize}
\subsection{Определение задачи обратного восстановления ЭЭГ}

\begin{wrapfigure}{r}{0.25\textwidth}
 \centering
 \begin{minipage}[b]{0.25\textwidth}
     \includegraphics[width=\textwidth]{../figures/head_surface_mesh.jpg}
     \caption{Поверхность головы $\Gamma$ }
     \label{fig: head_surface_mesh}
 \end{minipage}
 \hfill
 \begin{minipage}[b]{0.25\textwidth}
     \includegraphics[width=\textwidth]{../figures/head_mesh.jpg}  
     \caption{Рарез объема головы $\Omega$ }
     \label{fig: head_surface_mesh}
 \end{minipage}
 
\end{wrapfigure}
В общем виде обратная задача ЭЭГ представляет собой задачу восстановления скрытой переменной $s$ по наблюдаемым траекториям $X$. Различие подходов заключается в выборе априорного распределения и соответственно методам регуляризации. И методов восстановления траекторий по источникам. Также методы различаются в непрерывности либо дискретности пространства источников сигнала.
В общем виде задача формулируется следующим образом:

\begin{itemize}
    \item $\mathcal{D} = \{X_i\}_{i=1}^N$ - набор данных пространственно-временных рядов ЭЭГ, где $X_i = \chi(r, t): \mathbb{R}^3, \mathbb{R} \to \mathbb{R}^{K}$ - пространственно-временной ряд сигналов $K$.
    \item $\mathcal{S} = \{s_i(t)\}_{i=1}^M$ - конечный набор источников сигналов, зависит от выбранной физической модели, источники меняются во времени по своим неизвестным законам.  
\end{itemize}



Предполагается, что датчики $\chi$ расположены на границе $\Gamma = \delta \Omega$ области $\Omega$. В качестве $\Omega$ выступает весь объем головного мозга. Границей $\Gamma$ этой области является поверхность головы человека, на котором закреплены датчики. 

Найти

$D(\hat{s}|X, \mathcal{D})$ - суперпозиция пространственно-временных рядов ЭЭГ $X_i$ по $M$ источникам

$G(\hat{X}(t+1)|X(t), \hat{s}, \mathcal{D})$ - восстановление пространственно-временных рядов ЭЭГ $X_i$ на основе $s$ источника

Критерий

MSE - ошибка предсказаний рядов $\|\hat{X} - {X}\|_2$

Задача нахождения G называется прямой задачей EEG. Предполагается, что решение прямой задачи есть и находится вычислительно проще и не требует решения задачи оптимизации, чем решение обратной модели, например, считается аналитически. Важным отличием прямой задачи является единственность решении прямой задачи, в отличие от обратной В работе модель G задается интегрированием уравнений Максвелла.

В работе пространство источников считается непрерывным, также непрерывными являются поля. При этом учитывается, что наблюдаются сигналы с конечного числа датчиков, расположенных неравномерно по сетке $\delta \Omega$, данные от которых зашумлены.

\section{Модель сигнала ЭЭГ}

\begin{figure}[H]
\centerline{
    \xymatrix{
    \chi_i: (N_x \times N_y \times N_z \times T)  \ar@{->}[rr]^-{D(\hat{s}|X, \mathcal{D})} && s \ar@{->}[rr]^-{G(\hat{X}|X(t), \hat{s}, \mathcal{D})} && \hat{\mathbf{X}}
    }
}
\caption{Модель сигнала ЭЭГ}
\label{fig: LNN}
\end{figure}

Формально пространственно-временной ряд $\chi$ можно записать как ряд поля $E(\mathbf{r})_t$. Таким образом задача состоит в нахождении в источниках поля $E$. 

В работе предлагается использовать уравнения в частных производных для восстановления динамики поля. В общем виде это можно записать так
$$A\left(s, r, t, E, \frac{\partial E}{\partial{\mathbf{r}}}, \frac{\partial E}{\partial{t}}, ... \right)= 0.$$
\section{Физико информированный подход}
Для того чтобы внести физическую информацию о системе необходимо записать физический закон по которому изменяется система в представленной форме.

Рассматриваемое нами поле по сути является физическим электромагнитным полем с добавленным шумом от различных помех. Поэтому достаточно ограничиться электромагнитными физическими системами.

Наиболее общим подходом для электромагнитных будет использовать уравнения Максвелла
\begin{itemize}
    \item Для восстановления поля $E$ по известному значению по контуру. используем метод зеркальных отображений Рисунок~\ref{fig: trajectory3}. Нейросеть $D$ восстанавливает внутренние токи $j$и заряды $q$ по ЭЭГ $E$. 
    \item По теореме единственности есть только один такой набор для результатов ЭЭГ. 
    \item По внутренним токам и зарядам и уравнений Максвелла получается PDE на поле $\hat{E}$
    \item С помощью nPDE данные уравнения решаются и восстанавливается поле $\hat{E}$
\end{itemize}

Критерий для задачи восстановления поля
$$\| \hat{E} -  E\|_2^2$$

\begin{figure}[H]
 \centering
 \includegraphics[scale = 0.6]{../figures/image109.jpg}
 \caption{Метод зеркальных изображений}
 \label{fig: trajectory3}
\end{figure}

Уравнения Максвелла
\begin{itemize}
    \item $\nabla \cdot \mathbf{D} = 4\pi \rho$
    \item $\nabla \cdot \mathbf{B} = 0$
    \item $\nabla \times \mathbf{E} = -\frac{1}{c} \frac{\partial \mathbf{B}}{\partial t}$
    \item $\nabla \times \mathbf{H} = \frac{4\pi}{c} \mathbf{j} + \frac{1}{c} \frac{\partial \mathbf{D}}{\partial t}$
\end{itemize}
Где $\mathbf{j}$ - плотность внутренних токов в мозгу, $\rho$ плотность внутренних зарядов в мозгу.

Нейросеть $D(\hat{s}|X)$ восстанавливает источники $s: \mathbf{j}, \rho$ электромагнитного поля: плотность токов $\mathbf{j}$ и плотность заряда $\rho$

По ним составляется система уравнений в частных производных с помощью уравнений Максвелла.
Решения уравнений Максвелла ищется с помощью nPDE\cite{Zubov2021}

\subsection{Регуляризация энергии и её Монтекарло-аппроксимация}
\label{subsec:regularization}
В работе в качестве регуляризации предлагается использовать энергию электрического поля, формально задаваемую как
\[ E = \int_{\Omega} \phi(r)\rho(r)\,dr. \]
Практически полезна её дискретная аппроксимация на выборке точек в области $\Omega$, что ведёт к вычислительно эффективным регуляризаторам при обучении сетей.

\begin{lemma}[MC-эквивалентность энергии и среднего произведения $\phi\rho$]
Пусть $\Omega\subset\mathbb{R}^d$ — измеримое множество с конечным объёмом $|\Omega|>0$. Пусть функции $\phi,\rho:\Omega\to\mathbb{R}$ такие, что $\phi\rho\in L^1(\Omega)$. Пусть $r_1,\dots,r_K$ быть i.i.d. точками, выбранными по вероятностной мере $\mu$ на $\Omega$ с плотностью $p(r)>0$ почти всюду (в частности, допускается равномерная выборка $p\equiv1/|\Omega|$). Определим
\[ E:=\int_{\Omega}\phi(r)\rho(r)\,dr,\qquad \hat E_K:=\frac{1}{K}\sum_{i=1}^K w(r_i)\phi(r_i)\rho(r_i),\quad w(r)=\frac{1}{p(r)}.\]
Тогда выполняются следующие утверждения:
\begin{enumerate}
\item (Несмещённость) $\mathbb{E}[\hat E_K]=E$.
\item (Сходимость) $\hat E_K\xrightarrow{a.s.}E$ при $K\to\infty$; более того, при $\mathbb{E}[|w\phi\rho|]<\infty$ имеем $\hat E_K\to E$ в $L^1$.
\item (Случайная погрешность) Если дополнительно $w\phi\rho\in L^2(\mu)$, то по центральной предельной теореме
\[ \sqrt{K}(\hat E_K-E)\xrightarrow{d}\mathcal{N}(0,\sigma^2),\qquad \sigma^2=\mathrm{Var}_\mu(w\phi\rho). \]
Следовательно, типичный масштаб ошибки равен $O_p(1/\sqrt{K})$.
\item (Оценка хвостов) Если $|w(r)\phi(r)\rho(r)|\le M$ почти наверное, то по неравенству Хёффдинга для суммы независимых ограниченных слагаемых
\[ \mathbb{P}(|\hat E_K-E|\ge\varepsilon)\le 2\exp\Big(-\frac{2K\varepsilon^2}{M^2}\Big). \]
\end{enumerate}
\end{lemma}

\begin{proof}
Пункт 1: 
\[ \mathbb{E}[\hat E_K]=\mathbb{E}\Big[\frac{1}{K}\sum_{i=1}^K w(r_i)\phi(r_i)\rho(r_i)\Big]=\mathbb{E}[w(r)\phi(r)\rho(r)]=\int_{\Omega} \phi(r)\rho(r)\,dr=E, \]
где на втором шаге использована линейность матожидания и идентичность распределения сэмплов (см. \cite{Billingsley1995}).

Пункт 2: из предположения $\mathbb{E}[|w\phi\rho|]<\infty$ и закона больших чисел для i.i.d. последовательностей следует сходимость почти наверное $\hat E_K\to E$ и сходимость в среднем (при дополнительных условиях) --- в $L^1$ (см. \cite{Billingsley1995}).

Пункт 3: при конечной дисперсии $\mathrm{Var}_\mu(w\phi\rho)=\sigma^2<\infty$ применима центральная предельная теорема, которая даёт сходимость распределений к нормальной с указанной дисперсией (см. \cite{VanDerVaart1998}).

Пункт 4: если слагаемые ограничены по модулю $M$, то применима версия неравенства Хёффдинга для сумм i.i.d. ограниченных величин, откуда и следует указанная оценка (см. \cite{Hoeffding1963}).
\end{proof}

\begin{corollary}
Для фиксированного множителя регуляризации $\lambda>0$ замена непрерывного регуляризатора $\lambda E$ на дискретную версию $\lambda\hat E_K$ даёт состоятельную аппроксимацию: $\hat E_K\to E$ при $K\to\infty$ и погрешность порядка $O_p(1/\sqrt{K})$.
\end{corollary}

\begin{remark}
Аппроксимация вида $\frac{1}{K}\sum_i\phi(r_i)\rho(r_i)$ (для $p\equiv 1/|\Omega|$) даёт простую и эффективную реализацию регуляризатора во время стохастического градиентного спуска и уменьшает вычислительные затраты по сравнению с оценкой полной энергии по сетке.
\end{remark}

\section{Базовый эксперимент с линейной моделью}
В качестве эксперимента с линейным зависимостями между элементами предлагается рассмотреть следующую работу \cite{Palummo2024}. В отличие от ICA в нем рассматриваются векторное пространство функций. Рассматривается базис из функций $\psi$, по которому и раскладывается поле.

\begin{figure}[H]
 \centering
 \includegraphics[scale = 0.37]{../figures/lake_1996.png}
 \caption{Данные по температуре поверхности озера за 1996 год. Слева наблюдаемые, справа предсказанные.}
 \label{fig: trajectory3}
\end{figure}

\begin{figure}[H]
 \centering
 \includegraphics[scale = 0.37]{../figures/lake_2011.png}
 \caption{Данные по температуре поверхности озера за 2011 год. Слева наблюдаемые, справа предсказанные.}
 \label{fig: trajectory3}
\end{figure}
\section{Датасеты}
https://data.mrc.ox.ac.uk/data-set/simulated-eeg-data-generator

https://zenodo.org/records/2348892

https://ieee-dataport.org/documents/eeg-datasets-different-levels-fatigue-personal-identification

https://www.kaggle.com/datasets/harunshimanto/epileptic-seizure-recognition

https://www.kaggle.com/datasets/wanghaohan/confused-eeg

\section{Экспериментальные результаты}

В этом разделе мы описываем два основных эксперимента: (1) оценка на реальных данных проекта Brunton et al. (\texttt{brunton\_uw\_bio}), и (2) синтетический эксперимент на моделируемой овальной голове с 2 источниками и частотой дискретизации 50~Гц. Для оценки используем метрики: MSE на датчиках (Sensor MSE) и MSE для полей/источников в глубине (Deep MSE). Таблицы ниже содержат плейсхолдеры для занесения численных значений.

\subsection{Эксперимент 1 — реальные данные (Brunton)}
\label{subsec:exp1}
Данные: набор EEG из \texttt{brunton\_uw\_bio}. Предобработка: приведение к единой частоте выборки, удаление лишних каналов, приведение к общему усреднённому отсылочному каналу и фильтрация полосы 1–None~Hz, см. код в разделе \texttt{datasets/brunton}. Модель обучалась для восстановления потенциала на датчиках и внутренних полей по предложенной nPDE‑архитектуре; в качестве baselines использовалась реализация LORETA/Loretta.

Результаты (плейсхолдеры):
\begin{table}[h]
\centering
\begin{tabular}{lcc}
\hline
Метод & Sensor MSE & Примечания \\
\hline
Наш метод & --- &  \\
Loretta & --- &  \\
\hline
\end{tabular}
\caption{Сравнение по датчикам на Brunton (заполните значения)}
\end{table}

\subsection{Эксперимент 2 — синтетические данные}
\label{subsec:exp2}
Модель: синтетическая овальная оболочка головы, $n=16$ датчиков, дискретизация 50~Гц, два независимых источника во внутренней области. На рис. \ref{fig:synthetic_head_views} приведена визуализация используемой конфигурации датчиков и головы.
\begin{figure}[h]
\centering
\includegraphics[width=0.6\textwidth]{../figures/synthetic_head_views.png}
\caption{Визуализация синтетической головы и положения датчиков.}
\label{fig:synthetic_head_views}
\end{figure}

Результаты (плейсхолдеры):
\begin{table}[h]
\centering
\begin{tabular}{lccr}
\hline
Метод & Sensor MSE & Deep MSE & Примечания \\
\hline
Наш метод & --- & --- &  \\
Loretta & --- & --- &  \\
\hline
\end{tabular}
\caption{Сравнение по датчикам и глубине (заполните значения)}
\end{table}

\subsection{Сравнение методов регуляризации}
\label{subsec:reg_comp}
Мы сравниваем следующие подходы к регуляризации: 
\begin{itemize}
    \item Energy (нормальная форма $\int\phi\rho$) и её MC-аппроксимация (наш предложенный метод), 
    \item Tikhonov/L2 (гладкость в пространстве), 
    \item L1 / разреженность источников (sparse priors),
    \item Total Variation / дивергентные штрафы (для сохранения структур),
    \item Bayesian-approaches / KL-телескопирование при использовании априорных распределений источников.
\end{itemize}
Ключевые наблюдения из экспериментов:
\begin{itemize}
    \item На реальных данных (Brunton) MSE на датчиках у методов примерно сопоставимы, поэтому судить о качестве только по Sensor MSE нельзя.
    \item На синтетике наше регуляризованное решение даёт заметно лучшее восстановление в глубине (Deep MSE), при примерно равных показателях на датчиках.
    \item Вывод: для объективного сравнения методов рекомендуется использовать составной протокол оценки: Sensor MSE (если доступны датчики), Deep/Source MSE (на синтетике), а также метрики качества на побочных задачах (см. ниже).
\end{itemize}

\subsection{Рекомендация: сравнивать по побочным задачам}
Практически полезным критерием сравнения является оценка восстановленных источников в контексте downstream-задач: 
\begin{enumerate}
    \item Регрессия речевого/моторного ряда или прогноз временного ряда — измерять RMSE/MAE на валидационном наборе.
    \item Классификация действий (motor tasks) — измерять accuracy/F1 при обучении модели на восстановленных источниках.
    \item Устойчивость к шуму и интерполяция (оценивать по ресемплингу и добавлению искусственного шума).
\end{enumerate}
Эти показатели, дополненные оценкой неопределённости восстановления (bootstrap/MC dropout), дают более информативную картину чем сравнение только по MSE на датчиках.

\section{Базовый эксперимент по восстановлению поля заряда в пространстве}

В эксперименте рассматривается задача восстановления поля от заряда в трехмерном пространстве. Несмотря на простоту задачи ее решение совпадает с решением прямой задачи для ЭЭг, которая требуется для решения обратной задачи ЭЭГ.

\subsection{Общая формулировка задачи}
Формально дана задача Коши системы уравнения в частных производных следующего вида

\begin{equation}
\left\{ \begin{aligned} 
  \nabla \cdot \mathbf{D} = 4\pi \rho \\
    \nabla \cdot \mathbf{B} = 0 \\
    \nabla \times \mathbf{E} = -\frac{1}{c} \frac{\partial \mathbf{B}}{\partial t} \\
    \nabla \times \mathbf{H} = \frac{4\pi}{c} \mathbf{j} + \frac{1}{c} \frac{\partial \mathbf{D}}{\partial t} \\
\end{aligned} \right.
\end{equation}

Где плотность заряда представлена следующим образом
$$\rho(\mathbf{r}) = \delta(\mathbf{r})$$
Заметим, что такая функция не удовлетворяет условию Липшица и даже не является вещественной функцией.

Ток в системе отсутствует:
$$\mathbf{j} = 0$$

Со следующими граничными условиями
$$\mathbf{E}(\infty) = 0, \quad \mathbf{B}(\infty) = 0$$

В такой постановке задача численно не решается и даже не выполняются условия для существования и единственности решения. Поэтому используется следующая постановка задачи, которая в пределе стремится к исходной 
\subsection{Приближенное решение задачи}
В качестве плотности распределения заряда берется функция, сферически симметричная $f(\mathbf{r}) = f(r)$ и Липшицева. В работе рассматривается нормальное распределения с $\Sigma = \sigma \cdot \mathbf{I}$ и $\mathbf{\mu} = 0$. 
Данной выбор свойств функции обусловлен следующими обстоятельствами. Для теоремы существования и единственности необходимо и достаточно, чтобы функция была Липшецева. А согласно теореме Гаусса поле в точке $\mathbf{r}$ будет отличаться в $\int_{r' \leq r} f(\mathbf{r'}) d\mathbf{r'}$, таким образом, если вся плотность заряда будет сосредоточена в сфере радиусом $R$, то вне этой сферы поле не будет отличаться вовсе

В качестве граничных условий выберем радиус равный $R_\text{limit} = 2.2$. Он выбирается из расчета того, когда модуль поля будет меньше чем значение машинной точности.

$$\mathbf{E}(\mathbf{r}| r = R_\text{limit}) = 0, \quad \mathbf{B}(\mathbf{r}| r = R_\text{limit}) = 0$$

\subsection{Вид функции плотности распределения заряда в зависимости от $\sigma$}

\begin{figure}[H]
 \centering
 \includegraphics[scale = 0.2]{../figures/plot_rho_0.1.png}
 \caption{Плотность электрического заряда $\rho(\mathbf{r})$ для $\sigma = 0.1$}
 \label{fig: rho_0_1}
\end{figure}

\begin{figure}[H]
 \centering
 \includegraphics[scale = 0.2]{../figures/plot_rho_0.02.png}
 \caption{Плотность электрического заряда $\rho(\mathbf{r})$ для $\sigma = 0.02$}
 \label{fig: rho_0_02}
\end{figure}

\begin{figure}[H]
 \centering
 \includegraphics[scale = 0.2]{../figures/plot_rho_0.005.png}
 \caption{Плотность электрического заряда $\rho(\mathbf{r})$ для $\sigma = 0.005$}
 \label{fig: rho_0_005}
\end{figure}

\subsection{Сравнения численного решения для проекции электрического поля $E_x$ в зависимости от плотности распределения заряда $\sigma$}

\begin{figure}[H]
 \centering
 \includegraphics[scale = 0.37]../figures/{plot_Ex_0.1.png}
 \caption{Проекция электрического поля на ось Ox для $\sigma = 0.1$}
 \label{fig: E_x_0_1}
\end{figure}

\begin{figure}[H]
 \centering
 \includegraphics[scale = 0.37]{../figures/plot_Ex_0.02.png}
 \caption{Проекция электрического поля на ось Ox для $\sigma = 0.02$}
 \label{fig: E_x_0_02}
\end{figure}

\begin{figure}[H]
 \centering
 \includegraphics[scale = 0.8]{../figures/plot_Ex_0.005.png}
 \caption{Проекция электрического поля на ось Ox для $\sigma = 0.005$}
 \label{fig: E_x_0_005}
\end{figure}

\begin{figure}[H]
 \centering
 \includegraphics[scale = 0.37]{../figures/plot_Ex_0.0001.png}
 \caption{Проекция электрического поля на ось Ox для $\sigma = 0.0001$}
 \label{fig: E_x_0_0001}
\end{figure}

Заметим, что маленькие значения $\sigma$ дают плохие решения также как и большие. Это связано с тем, что для численного решения вероятность рассмотрения области фазового пространства равна его объему. Соответственно если носитель функции $\rho$ имеет объем сильно меньше, чем домен, то вероятность взять точку из него примерно равна 0. Таким образом решение будет стремиться к решению с задачи с нулевой плотностью, а не исходной.

Этот факт говорит еще, что домен не должен быть слишком большим, иначе численные решение также будет стремиться к нулю. Если требуется рассмотреть большие домены, то их либо надо делить на мелкие, либо находить приближенные аналитические решения

\subsection{Проекции электрического поля $E_y$, $E_Z$ на другие оси для $\sigma = 0.02$}

\begin{figure}[H]
 \centering
 \includegraphics[scale = 0.37]{../figures/plot_Ey_0.02.png}
 \caption{Проекция электрического поля на ось Oy для $\sigma = 0.02$}
 \label{fig: E_y_0_02}
\end{figure}

\begin{figure}[H]
 \centering
 \includegraphics[scale = 0.37]{../figures/plot_Ez_0.02.png}
 \caption{Проекция электрического поля на ось Oz для $\sigma = 0.02$}
 \label{fig: E_z_0_02}
\end{figure}

\bibliography{npde-bibliography}

\end{document}
